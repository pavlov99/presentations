\documentclass[unicode, notheorems, aspectratio=169]{beamer}
% Aspect ratio: https://tex.stackexchange.com/questions/14336/latex-beamer-presentation-package-169-aspect-ratio#answer-14339
\usepackage{etex}  % Should be second line. Otherwise tikz raises error.
\hypersetup{pdfpagemode=FullScreen}

% If you have more than three sections or more than three subsections in at least one section,
% you might want to use the [compress] switch. In this case, only the current (sub-) section
% is displayed in the header and not the full overview.
\mode<presentation>
{
  \usetheme{Madrid}
  \usecolortheme{seahorse}
}

% https://tex.stackexchange.com/questions/106789/why-does-usepackaget2afontenc-take-over
\usepackage[T2A,T1]{fontenc}
\usepackage[english]{babel}
\usepackage{amsthm}
\usepackage[noend]{algorithmic}
\usepackage{algorithm}
\usepackage[all]{xy} % for graph plotting
% \usepackage{times}
\usepackage{tikz}
\usetikzlibrary{arrows,shapes,backgrounds}
\usepackage{textcomp} % euro sign
\usepackage[official]{eurosym}
\usepackage{ulem} % strike
\usepackage{color} % use colors (in minted)

% tables
\usepackage{booktabs}
\usepackage{multirow}

%% Code highlight
% Doc: http://ftp.yzu.edu.tw/CTAN/macros/latex/contrib/listings/listings.pdf
\usepackage{listings}
\usepackage{minted}
\definecolor{LightGray}{rgb}{0.95,0.95,0.95}
\setminted{bgcolor=LightGray}
% \usemintedstyle{native}
% ftp://ftp.dante.de/tex-archive/macros/latex/contrib/minted/minted.pdf
% Note: to use minted, install pygments http://pygments.org/ and add "-shell-escape" flag to latex command.
% Minted doc: http://ftp.yzu.edu.tw/CTAN/macros/latex/contrib/minted/minted.pdf
%% END: Code highlight

\usepackage{helvet}

\title[Boosting command line: python + awk]{Boosting command line experience}
\subtitle{Python meets AWK}
\author[Kirill Pavlov <k@p99.io>]{Kirill Pavlov}
\institute[]{Technical Recruiter, Terminal 1}
\titlegraphic{\includegraphics[width=3cm]{./images/pycon-logo}}
\date{14:40 -- 15:10, November 05, 2017}

\definecolor{t1_light}{RGB}{38,47,112}
\definecolor{t1_dark}{RGB}{39,47,56}
\definecolor{t1_blue}{RGB}{50,121,235}

\setbeamertemplate{title page}[default][colsep=-4bp,rounded=true]
\setbeamertemplate{itemize items}[default]
\setbeamertemplate{enumerate items}[default]

\setbeamercolor*{palette primary}{use=structure,fg=white,bg=t1_light}
\setbeamercolor*{palette secondary}{use=structure,fg=white,bg=t1_dark}
\setbeamercolor*{palette tertiary}{use=structure,fg=white,bg=t1_blue}

\setbeamercolor{itemize item}{fg=t1_blue}
\setbeamercolor{itemize subitem}{fg=t1_blue}
\setbeamercolor{enumerate item}{fg=t1_dark}
\setbeamercolor{enumerate subitem}{fg=t1_dark}

\setbeamertemplate{section in toc}{%
  {\color{t1_dark}\inserttocsectionnumber.}~\inserttocsection}
\setbeamercolor{subsection in toc}{bg=white,fg=structure}
\setbeamertemplate{subsection in toc}{%
  \hspace{1.2em}{\color{t1_dark}\rule[0.3ex]{3pt}{3pt}}~\inserttocsubsection\par}
  
% Set beamer blocks
\setbeamertemplate{blocks}[rounded][shadow=false]
\addtobeamertemplate{block begin}{\pgfsetfillopacity{0.8}}{\pgfsetfillopacity{1}}

\setbeamercolor*{block title example}{fg=white,bg= t1_light}
\setbeamercolor*{block body example}{fg=white,bg= t1_dark}

% remove page navigation.
\beamertemplatenavigationsymbolsempty

% Set link colours
\definecolor{links}{HTML}{2A1B81}
\hypersetup{colorlinks,linkcolor=,urlcolor=links}

\begin{document}
% For every picture that defines or uses external nodes, you'll have to
% apply the 'remember picture' style. To avoid some typing, we'll apply
% the style to all pictures.
\tikzstyle{every picture}+=[remember picture]

% By default all math in TikZ nodes are set in inline mode. Change this to
% displaystyle so that we don't get small fractions.
\everymath{\displaystyle}

% This presentation shows ability of spark to work with data.
% It covers data processing, pipelining, feature engineering and algorithms available in Spark.

\begin{frame}
\titlepage
\end{frame}

\begin{frame}{README.md}
\begin{itemize}
\item A lot of Python/AWK examples here.
\item Source code and slides \href{https://github.com/pavlov99/presentations/tree/master/2017-11-05-pycon}{available online}.
\item At the end: build a stock trading system and check NYSE:CS.
\end{itemize}
\end{frame}

\begin{frame}{Table of content}
	% \tableofcontents[currentsection]
    \tableofcontents
\end{frame}

\section{Problem Background}
\begin{frame}{Table of content}
	\tableofcontents[currentsection]
\end{frame}

\begin{frame}{Background}
\begin{itemize}
\item \href{https://yandex.com}{Yandex}, year 2010. Hadoop was not widly adopted.
\item 10Gb of archived ads data daily: \textit{time}, \textit{ad\_id}, \textit{site\_id}, \textit{clicks}.
\item Task: daily data aggregation (simple functions: group by, sum, join) and feature generation for further machine learning classification.
\item Solution: released a set of command line scripts.
\end{itemize}
\end{frame}

\begin{frame}{Example}
This presentation uses UCI machine learning \href{http://archive.ics.uci.edu/ml/datasets/HIGGS}{Higgs boson} data: 11M objects, 28 attributes, 7.5Gb unarchived. 
\vfill
\textbf{Questions:}
\begin{enumerate}
\item What is the maximum value of \textit{lepton\_eta}?
\item What is the average value of \textit{lepton\_phi} by class 0 and 1?
\item Filter objects with \textit{m\_jj} > 0.75 (8.9M objects) and sort them by \textit{m\_wbb}.
\end{enumerate}
\vfill
\textbf{Solutions:}
\begin{enumerate}
\item In-memory Python with Pandas.
\item Databse SQL queries (PostgreSQL and Docker).
\item Command line with AWK.
\end{enumerate}
\end{frame}

\begin{frame}[noframenumbering]
	\begin{center}
		{\huge Demo Time}
	\end{center}
\end{frame}

\begin{frame}{Reality Check}
It's not as agile as it seems. You work inside the company network.
\vfill
\begin{enumerate}
\item You \textbf{don't have sudo} rights and your admin does not want to install anything for you. Like no database or user privileges, etc.
\item The \textbf{server does not have GitHub/Internet access} and the only deployment possible is Java JARs or C/C++/etc. So, no NodeJS/Python packages. And of course no R/Matlab/Excel.
\item Get better at command line tools ;)
\end{enumerate}
\end{frame}

\section{AWK Bootcamp in 5 min}
\begin{frame}{Table of content}
	\tableofcontents[currentsection]
\end{frame}
\begin{frame}{Basic concepts}
\begin{enumerate}
\item \href{https://en.wikipedia.org/wiki/AWK}{AWK}\footnotemark --- language for streaming columnar data processing. Standard in unix-like OS.
\item Actual AWK is outdated, use mawk (fast) or gawk (flexible).
\item Limited data structures: strings, \textbf{associative arrays (hash maps)} and regexps.
\item Built-in variables:
	\begin{itemize}
	\item \$1, \$2, \dots (\$0 is entire record)
	\item NR - number of processe lines (records)
	\item NF - number of columns (fields)
	\end{itemize}
\item Use vars without declaration. Default values are 0s. One liners. \textbf{Hipster friendly}.
\end{enumerate}
\footnotetext[1]{
\href{http://www.grymoire.com/Unix/Awk.html}{Tutorial} by Bruce Barnett. Careful, he \href{http://www.grymoire.com/Unix/CshTop10.txt}{writes his blog in txt}}
\end{frame}

\begin{frame}[fragile]{AWK Example 1}
1.1 How many speakers does PyCon have (awk + sort)?
\begin{minted}{bash}
cat pycon-schedule.csv \
  | awk -F, '{if($4) print $4}' \
  | tail -n+2 | sort \
  | uniq | wc -l
\end{minted}
\vfill
1.2 How many speakers does PyCon have (tawk + tsort)?
\begin{minted}{bash}
cat pycon-schedule.tsv \
  | tawk -o 'speaker' -f 'speaker' \
  | tsrt -N \
  | uniq | wc -l
\end{minted}
\end{frame}

\begin{frame}[fragile]{AWK Example 2}
2. What is the most popular word in PyCon topics?
\begin{minted}{bash}
cut -d, -f5 pycon-schedule.csv \
  | tail -n+2 \
  | awk '{for(i=1; i<=NF; i++) w[\$i]++}END{for(i in w) print w[i], i}' \
  | sort -r | head
\end{minted}
\pause
\vfill
Python! (who could have guessed?)
\end{frame}

\begin{frame}[fragile]{AWK Example 3}
3. Filter unique talks (one talk at a time)
\begin{minted}{bash}
cat pycon-schedule.tsv \
  | tawk -a -f 'speaker' \
    -o 'key=timeslot if day == 1 else gensub("-", "+", "g", timeslot)' \
  | tgrp -k key \
    -g 'day=FIRST(day)' -g 'timeslot=FIRST(timeslot)' \
    -g 'venue=FIRST(venue)' -g 'speaker=FIRST(speaker)' \
    -g 'title=FIRST(title)' -g 'sessions=COUNT()' \
  | tawk -o 'day;timeslot;venue;speaker;title' -f 'sessions==1'
\end{minted}
\end{frame}

\section{Tabtools architecture and features}
\begin{frame}{Table of content}
	\tableofcontents[currentsection]
\end{frame}

\begin{frame}[fragile]{Basic concepts}
\begin{enumerate}
\item Special files format: tsv + header (meta information). Easy to convert and autogenerate headers.
\begin{minted}{bash}
# Date      Open    High   Low   Close  Volume
2014-02-21  84.35   84.45  83.9  83.45  17275.0
\end{minted}
\item Python script manages file descriptors headers, convert column names to column numbers and executes command line command, e.g. cat/tail/sort.
\item Heavy lifting goes to awk: tawk (map) and tgrp (map-reduce).
\item Based on command line expressions, it generates awk command and executes it with incoming stream.
\item Visual sugar: tpretty and tplot.
\end{enumerate}
\end{frame}

\begin{frame}{Under the hood: FileList and stream management}
\begin{center}
	\includegraphics[width=\textwidth]{./images/file-list}
\end{center}
\end{frame}

\begin{frame}[fragile]{Under the hood: Argument parsing}
Consider bash command with two imput files and sort by one key:
\begin{minted}{bash}
cat file1.tsv | tsrt -k field1 - file2.tsv
\end{minted}
\vfill
\begin{center}
	\includegraphics[height=.7\textheight]{./images/argument-parsing}
\end{center}
\end{frame}

\begin{frame}{Under the hood: AWK program expressions}
\begin{center}
	\includegraphics[width=.9\textwidth]{./images/awk-program-expressions}
\end{center}
\end{frame}

\begin{frame}{Under the hood: Expression context}
\begin{center}
	\includegraphics[width=.7\textwidth]{./images/expression-context}
\end{center}
\end{frame}

\begin{frame}{Homework}
Given stream of data: 1, 2, 3, $\dots$ Write a program to calculate:
\begin{enumerate}
\item Streaming average S, such as S1=1, S2=(1+2)/2 S3=(1+2+3)/3
\item Running average A with window 3, such as A1=1, A2=S2, A4=(4+5+6)/3
\item Running maximum M with window 3, such as M1=1, M4=max(4, 5, 6)=6
\end{enumerate}
\end{frame}

\begin{frame}{Features}
\begin{enumerate}
\item Streaming expressions: parametrized running/total sum/average/maximum\footnotemark.
\item Aggregators: first, last, min, max, count.
\item Modules: \href{https://en.wikipedia.org/wiki/Double-ended_queue}{deque}.
\item Build to self-contained ~2k LOC portable python (2.7, 3.3+) scripts.
\item All together: zero-configuration extensible sql in command line. It is readable and faster than a generic python/cython code (even after shedskin) and perl. 
\end{enumerate}

\footnotetext[2]{moving maximum in \href{https://www.quora.com/How-can-I-solve-this-array-moving-window-max-problem-in-linear-time}{linear time} with \href{https://gist.github.com/pavlov99/c7b7bcd7913ffa72d651}{deque} implemented on top of awk associative arrays.}
\end{frame}

\begin{frame}{Solutions comparison}
Dell xps 15, 16Gb RAM, 8 CPUs:
\vfill
\begin{tabular}{ |l|r|r|r|r|r| } 
  \hline
  & \textbf{Python} & \textbf{PostgreSQL} & \textbf{gawk} & \textbf{mawk} & \textbf{tabtools} \\ \hline
  Read time & 104.4 & 180.3 & 0 & 0 & 0 \\ \hline
  Q1: "max" time & 0 & 15.2 & 22.8 & 12.2 & 12.8 \\ \hline
  Q2: "group + avg" time & 0 & 5.8 & 30.5 & 12.6 & 26.6\footnotemark \\ \hline
  Q3: "filter + sort" time & 21.3 & 33.6 & 174.2 & 36.3 & 33.5 \\ \hline
  \textbf{Total, sec.} & 125.7 & 243.9 & 227.5 & \alert{61.1} & \alert{72.9} \\ \hline
\end{tabular}
\footnotetext[3]{Uses $\Omega(n \log(n))$ complexity instead of $\Omega(n)$. Could be improved.}
\end{frame}

\section{Stock example with NYSE:CS}
\begin{frame}{Table of content}
	\tableofcontents[currentsection]
\end{frame}

\begin{frame}[fragile]{Data description}
Credit Suisse (NYSE:CS) daily stock data from \href{https://finance.yahoo.com/quote/CS/history?period1=800553600&period2=1508515200&interval=1d&filter=history&frequency=1d}{Yahoo Finance}: 'CS.csv' + 'cs.tsv'.
\vfill
\begin{minted}{bash}
cat cs.tsv | tgrp \
    -k "Week=strftime(\"%U\", DateEpoch(Date))" \
    -g "Date=FIRST(Date)" \
    -g "Open=FIRST(Open)" \
    -g "High=MAX(High)" \
    -g "Low=MIN(Low)" \
    -g "Close=LAST(Close)" \
    -g "Volume=SUM(Volume)" \
  | ttail \
  | tsrt -k Date:desc \
  | tpretty
\end{minted}
\end{frame}

\begin{frame}[noframenumbering]
	\begin{center}
		{\huge Demo Time}
	\end{center}
\end{frame}

\begin{frame}{Demo: metrics}
\begin{enumerate}
\item Moving Average for windown size 200 and 50.
\item Exponential moving average for window size 26 and 13.
\item \href{https://en.wikipedia.org/wiki/MACD}{MACD(26, 12, 9)} histogram.
\item \href{https://www.quora.com/How-can-I-solve-this-array-moving-window-max-problem-in-linear-time}{Moving maximum} and minimum for window size 14.
\item \href{https://www.fidelity.com/learning-center/trading-investing/technical-analysis/technical-indicator-guide/fast-stochastic}{Fast} and \href{https://www.fidelity.com/learning-center/trading-investing/technical-analysis/technical-indicator-guide/slow-stochastic}{Slow} Stochastics.
\end{enumerate}
\end{frame}

\begin{frame}{Demo: plot (expected and actual)}
\begin{center}
	\includegraphics[width=.5\textwidth]{./images/credit-suisse-stockcharts}
	\includegraphics[width=.5\textwidth]{./images/metrics}
\end{center}
\end{frame}

\begin{frame}
\begin{center}
{\huge Thank you!}

\vfill
Kirill Pavlov <k@p99.io>, Recruiter, \href{https://t1.gl/k}{Terminal 1}.

{\small
GitHub: \href{https://github.com/pavlov99}{@pavlov99} |
Presentation: \href{https://github.com/pavlov99/presentations/tree/master/2017-11-05-pycon}{2017-11-05-pycon} |
\href{https://github.com/pavlov99/tabtools}{tabtools} \\
\#jobs \href{https://t1.gl/k}{https://t1.gl/k}
}
\end{center}
\end{frame}

\end{document}
